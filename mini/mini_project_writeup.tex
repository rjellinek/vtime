% TEMPLATE for Usenix papers, specifically to meet requirements of
%  USENIX '05
% originally a template for producing IEEE-format articles using LaTeX.
%   written by Matthew Ward, CS Department, Worcester Polytechnic Institute.
% adapted by David Beazley for his excellent SWIG paper in Proceedings,
%   Tcl 96
% turned into a smartass generic template by De Clarke, with thanks to
%   both the above pioneers
% use at your own risk.  Complaints to /dev/null.
% make it two column with no page numbering, default is 10 point

% Munged by Fred Douglis <douglis@research.att.com> 10/97 to separate
% the .sty file from the LaTeX source template, so that people can
% more easily include the .sty file into an existing document.  Also
% changed to more closely follow the style guidelines as represented
% by the Word sample file. 

% Note that since 2010, USENIX does not require endnotes. If you want
% foot of page notes, don't include the endnotes package in the 
% usepackage command, below.

% This version uses the latex2e styles, not the very ancient 2.09 stuff.
\documentclass[letterpaper,twocolumn,10pt]{article}
\usepackage{usenix,epsfig,endnotes}
\begin{document}

%don't want date printed
\date{}

%make title bold and 14 pt font (Latex default is non-bold, 16 pt)
\title{\Large \bf The Effect of Virtualization in Timing Filesystem Operations}

%for single author (just remove % characters)
\author{
{\rm Your N.\ Here}\\
\and
{\rm Second Name}\\
% copy the following lines to add more authors
% \and
% {\rm Name}\\
%Name Institution
} % end author

\maketitle

% Use the following at camera-ready time to suppress page numbers.
% Comment it out when you first submit the paper for review.
\thispagestyle{empty}


\subsection*{Abstract}

\section{Introduction}

Features\endnote{Remember to use endnotes, not footnotes!} galore, plethora of promises.\\

\section{Introduction}

Now we're going to cite somebody.  Watch for the cite tag.
Here it comes~\cite{Chaum1981,Diffie1976}.  The tilde character (\~{})
in the source means a non-breaking space.  This way, your reference will
always be attached to the word that preceded it, instead of going to the
next line.

\section{Methodology}
Adam will get this.
\subsection{Environment}
\subsection{Buffer Size}
\subsection{Prefetching}
\subsection{File Cache}
\subsection{Inode Indirection}

\section{Results}
Rob will get this.
\subsection{Buffer Size}
\subsection{Prefetching}
\subsection{File Cache}
\subsection{Inode Indirection}

\section{Conclusions}

{\footnotesize \bibliographystyle{acm}
\bibliography{../common/bibliography}}

\theendnotes

\end{document}







